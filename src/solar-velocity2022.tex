% Note:
% - Could expand scope, fit the solar position above the midplane, orientation
%   of the plane, title "The Sun's Position and Velocity in the Galaxy"
% - Have to cite cosmohub...

\documentclass[RNAAS]{aastex631}

% Load common packages
% \usepackage{microtype}  % ALWAYS!
% \usepackage{amsmath}
% \usepackage{amsfonts}
% \usepackage{amssymb}
% \usepackage{booktabs}
% \usepackage{graphicx}
% % \usepackage{color}

\graphicspath{{figures/}}
% \definecolor{cbblue}{HTML}{3182bd}
% \usepackage{hyperref}
% \definecolor{linkcolor}{rgb}{0.02,0.35,0.55}
% \definecolor{citecolor}{rgb}{0.45,0.45,0.45}
% \hypersetup{colorlinks=true,linkcolor=linkcolor,citecolor=citecolor,
%             filecolor=linkcolor,urlcolor=linkcolor}
% \hypersetup{pageanchor=true}

\newcommand{\documentname}{\textsl{Article}}
\newcommand{\sectionname}{Section}
\renewcommand{\figurename}{Figure}
\newcommand{\equationname}{Equation}
\renewcommand{\tablename}{Table}

% Missions
\newcommand{\project}[1]{\textsl{#1}}

% Packages / projects / programming
\newcommand{\package}[1]{\textsl{#1}}
\newcommand{\acronym}[1]{{\small{#1}}}
\newcommand{\github}{\package{GitHub}}
\newcommand{\python}{\package{Python}}
\newcommand{\emcee}{\project{emcee}}

% Stats / probability
\newcommand{\given}{\,|\,}
\newcommand{\norm}{\mathcal{N}}
\newcommand{\pdf}{\textsl{pdf}}

% Maths
\newcommand{\dd}{\mathrm{d}}
\newcommand{\transpose}[1]{{#1}^{\mathsf{T}}}
\newcommand{\inverse}[1]{{#1}^{-1}}
\newcommand{\argmin}{\operatornamewithlimits{argmin}}
\newcommand{\mean}[1]{\left< #1 \right>}

% Non-scalar variables
\renewcommand{\vec}[1]{\ensuremath{\bs{#1}}}
\newcommand{\mat}[1]{\ensuremath{\mathbf{#1}}}

% Unit shortcuts
\newcommand{\msun}{\ensuremath{\mathrm{M}_\odot}}
\newcommand{\mjup}{\ensuremath{\mathrm{M}_{\mathrm{J}}}}
\newcommand{\kms}{\ensuremath{\mathrm{km}~\mathrm{s}^{-1}}}
\newcommand{\mps}{\ensuremath{\mathrm{m}~\mathrm{s}^{-1}}}
\newcommand{\pc}{\ensuremath{\mathrm{pc}}}
\newcommand{\kpc}{\ensuremath{\mathrm{kpc}}}
\newcommand{\kmskpc}{\ensuremath{\mathrm{km}~\mathrm{s}^{-1}~\mathrm{kpc}^{-1}}}
\newcommand{\dayd}{\ensuremath{\mathrm{d}}}
\newcommand{\yr}{\ensuremath{\mathrm{yr}}}
\newcommand{\AU}{\ensuremath{\mathrm{AU}}}
\newcommand{\Kel}{\ensuremath{\mathrm{K}}}

% Misc.
\newcommand{\bs}[1]{\boldsymbol{#1}}

% Astronomy
\newcommand{\DM}{{\rm DM}}
\newcommand{\feh}{\ensuremath{{[{\rm Fe}/{\rm H}]}}}
\newcommand{\mh}{\ensuremath{{[{\rm M}/{\rm H}]}}}
\newcommand{\df}{\acronym{DF}}
\newcommand{\logg}{\ensuremath{\log g}}
\newcommand{\Teff}{\ensuremath{T_{\textrm{eff}}}}
\newcommand{\vsini}{\ensuremath{v\,\sin i}}
\newcommand{\mtwomin}{\ensuremath{M_{2, {\rm min}}}}

% TO DO
\newcommand{\todo}[1]{{\color{red} TODO: #1}}

\newcommand{\gaia}{\textsl{Gaia}}
\newcommand{\dr}[1]{\acronym{DR}#1}
\newcommand{\apogee}{\acronym{APOGEE}}
\newcommand{\sdss}{\acronym{SDSS}}
\newcommand{\sdssiv}{\acronym{SDSS-IV}}
\newcommand{\thejoker}{\project{The~Joker}}

\newcommand{\sgrA}{\ensuremath{\textrm{Sgr}~\textrm{A}^{*}}}

\shorttitle{The Solar Velocity as of 2022}
\shortauthors{Price-Whelan \& Drimmel}

\begin{document}

\title{The Solar Position and Velocity in the Galaxy}

\author[0000-0003-0872-7098]{Adrian~M.~Price-Whelan}
\affiliation{Center for Computational Astrophysics, Flatiron Institute,
             162 Fifth Avenue, New York, NY 10010, USA}
\email{aprice-whelan@flatironinstitute.org}
\correspondingauthor{Adrian M. Price-Whelan}

% \author[0000-0002-1777-5502]{Ronald~Drimmel}
% \affiliation{Osservatorio Astrofisico di Torino—INAF, Pino Torinese, Italy}


\begin{abstract}\noindent
Sup
\end{abstract}

\section{Introduction} \label{sec:intro}

The (vector) position and velocity of the Sun within the Milky Way are
fundamental quantities used in many Galactic dynamics and Galactic astronomy
contexts.
For example, it is used to convert the observed, Heliocentric positions and
velocities of stars measured by the \gaia\ mission into Galactocentric
coordinates that are used to study the structure of the Galaxy
\cite[e.g.,][]{todo}, the kinematics of stars, clusters, and satellite galaxies
\cite[e.g.,][]{todo}, and as an initial step in computing Galactic orbits of
these objects \citep[e.g.,][]{}.
The solar position and velocity are also used to interpret observations of the
gas content of the Galaxy \citep{todo}, \todo{... other use cases or is this
TMI?}.

Historically, the solar velocity has often been expressed as the sum of two
components: (1) the velocity of the ``local standard of rest'' (LSR), which is
the velocity of a fictitious object on a circular orbit at the Sun's position,
and (2) the solar ``peculiar velocity,'' which is the velocity of the Sun
relative to the LSR.
This separation made sense because, again historically, the solar peculiar
velocity could be measured more precisely than the total velocity of the Sun due
to large uncertainties on the distance from the Sun to the Galactic center
$D_\odot$.

More recently, the interferometric \acronym{GRAVITY} instrument has enabled
incredibly precise measurements of the Sun--Galactic center distance $D_\odot$
through detailed modeling of the orbits of stars in the gravitational sphere of
influence of \sgrA\ (e.g., ``S2'').
As pointed out in \citet{Drimmel:2018}, these direct distance measurements mean
that it is now possible to obtain very precise estimates of the \emph{total}
solar velocity without having to distinguish between the LSR and peculiar
velocity of the sun.
In this \textit{Note}, we use the most recent (and most precise)
\acronym{GRAVITY} measurements of the distance to, and velocity of, \sgrA to
provide an updated value of the solar velocity with respect to the Galactic
center.


\section{The Solar Velocity} \label{sec:vel}

As is standard, we assume that \sgrA\ sits (at rest) at the Galactic center such
that the distance to \sgrA\ is equivalent to the distance to the Galactic
center, and such that the proper motion and radial velocity of \sgrA\ in a solar
system barycentric reference frame can be inverted to measure the velocity of
the Sun with respect to the Galactic center.



\bibliographystyle{aasjournal}
\bibliography{solar-velocity2022}

\end{document}
